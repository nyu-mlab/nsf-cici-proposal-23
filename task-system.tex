\section{Task 2: Integrating with existing state-of-the-art network systems}

From Task 1, we are able to characterize individual researchers' workloads from self-reports and from empirical data. An hypothetical characterization might look like: ``Dr. Alice Li from the Physics Department reguarly conducts high-bandwidth transfers with the Department of Energy,'' as illustrated in Figure~\ref{fig:dashboard}. This allows us to pre-populate the allow list with the DoE flow, so that the researcher, upon first using the dashboard, will not be starting from scratch.

In Task 2, we will describe how to implement the dashboard to integrate with the existing network system, how to continuously annotate traffic, and how to catch errors when there is a mistake either from the automatic annotations or from the researchers.


\paragraph{Implementing the dashboard to integrate with the control plane}
The dashboard shows a personalized view of network flows for every authenticated researcher. These flows are based on network measurement and analysis captured by Zeek [G] (Figure~\ref{fig:system}), which obtained the data by mirroring ports on various switches (such as [B] and [F]). It is normal that a research device may have dozens or even hundreds of concurrent flows. To prioritize which ones to show on the dashboard, we will check against the profiles obtained in Task 1. For example, in the case of the hypothetical physicist Dr. Alice Li, we will have known from Task 1a that she claims to be regularly transmitting data with the DoE, and from Task 1b that her highest bandwidth transferrs were indeed with the DoE based on network measurement. Based on this knowledge, the dashboard will prioritize showing DoE flows, so that she can decide whether to include the DoE flows into her allow list (Figure~\ref{fig:dashboard}).

Once a researcher creates an entry in the allow list (e.g., clicking the ``Apply'' button in Figure~\ref{fig:dashboard} will create an entry ``Alice Li to send at most 10,000 GB to DoE IP addresses or ASNs''), the dashboard [D] to transmit this allow list into the appropriate rules in the control plane [E]. Our design is meant to be hardware agnostic, as we would like our system to be deployed not only on NYU's HSRN but also other institutions' networks in general. In the case of NYU's network (which uses 8 Arista 7280DR3 switches with 24 400G ports in the core network), the dashboard will create a rule based on Arista EOS API [CITE], such that the existing DoE flows will be rerouted from the slow path ([B] to [C]) to the fast path ([B] to [F]), based on Figure~\ref{fig:system}. Our design will be modular in nature, so that the translation from user-defined rules into switch-readable rules will be encasuplated in different modules per vendor. For instance, if another high-speed network uses Cisco switches, the module will translate allow list rules into the appropriate Cisco APIs on the control plane [E] (which, in many cases, exists in the same device as the switch [B]); and if the network is a software defined network (SDN) based on P4 Integrated Network Stack, the module will provide the appropriate translations as well (in which case [E] will be an SDN controller).

% https://www.arista.com/assets/data/pdf/Whitepapers/Arista_eAPI_FINAL.pdf



\paragraph{Catching errors automatically}
Mistakes are possible. For example, a researcher may accidentally select a non-research flow or a flow they do not recognize. In this case, the allow list is too broad and may increase the attack surface.

We take two approaches to detect such errors. First, from Task 1, we already know each researcher's typical workload based on their self-reports (Task 1a) and network measurement (Task 1b). If a researcher includes a destination not known from Task 1 \textit{a priori}, the dashboard will alert the researcher. Using Figure~\ref{fig:dashboard} as an example, if the hypothetical physicist decides to add the Google flows to the allow list, the dashboard will ask, ``Are you sure to add Google to the allow lists? Your typical research workflow does not include Google.''

In our second approach to detect user errors, we plan to use an IDS that runs anomaly detection. To illustrate how this works, suppose a network flow, $f$ (such as email traffic) is originally not in the allow list and normally go through the slow path, i.e., from [A] to [B] and then to [C], per Figure~\ref{fig:system}. Somehow, the researcher accidentally adds $f$ to the allow list for security bypass (e.g., unintentional mouse clicks on the dashboard, Figure~\ref{fig:dashboard}). This action causes $f$ to be rerouted from [A] to [B] and then to [F].

In our system design, traffic from [F] is mirrored (e.g., through a network tap) to a server running Zeek [G]. By implementing anomaly detection algorithms such as Isolation Forest on the Zeek traffic logs, we hope to automatically detect $f$, which deviates from the researcher's typical research workload commonly observed on the fast path. Our anomaly detection does not run in-line (in contrast to traditional anomaly detection that runs in security middleboxes, such as [C] in Figure~\ref{fig:system}). The benefit of this technique is we impose mimimal performance overhead, as the port mirroring typically can be done at line speed. One limitation is that we might catch anomalies after the fact. Our preliminary implementation of the Isolation Forest takes about 30 seconds to detect an injected simulated anomaly; at 400 Gbps, this delay would translate into 1.5 TB exfiltrated, although we will explore techniques to reduce this delay even further. As anomaly detection is a well explored topic, we defer to the literature for implementation and fine-tuning, and will not discuss the details in this proposal. Our hope is that the chance of triggering the anomaly detection is small, as our approach is based on allow lists;  we will have alerted the researcher in the first place if the researcher decides to add a destination to the allow list that has not been seen before from Task 1.

Once we detect an anomaly, we automatically take two actions. First, our anomaly detection module [G] will manipulate the control plane [E] (e.g., through the API of the switch's operating system, or through an SDN controller in the case of a software-defined network), such that $f$ will be reverted back to the default slow path. At this point, $f$ will be rerouted from [A] to [B] and then to [C], where a traditional security appliance will inspect the packets and decide on further actions (e.g., blocking if the signature matches known malware activities). The second action we will take is to alert the researcher on the dashboard, e.g., through push notifications on the browser [D]. The researcher could decide to ignore this notification (in which case $f$ would stay on the slow path), or override this automated action (in which case $f$ will be added back to the allow list and rerouted through the fast path). In the case of a manual override, the anomaly detection algorithm will improve itself through reinforcement learning, so that $f$ will not be flagged as an anomaly for this researcher the next time.

To augment the Isolation Forest, we will explore collecting more endpoint data, such as operating system logs and server logs from researchers' devices [A]. We want to catch cases where the researcher's device itself is compromised (e.g., through malware on another network), the device commnunicates with the malware's command and control, and somehow the researcher is tricked into putting the malware traffic into the allow list (e.g., when the malware-contacted domain is visually similar to a legitimate research-related domain). Note that our threat model assumes no rogue researchers, because if a researcher were to turn rogue (i.e., an insider threat), even the traditional security appliance will have challenges in detecting such cases, especially when the rogue researcher relies on social engineering for attacks (beyond the scope of this proposal). We assume that the researcher's device could be compromised by malware while on another network. To prevent malware from modifying the dashboard, we will adopt the zero-trust model [CITE] and authenticate every modification to the allow list (e.g., through enterprise multi-factor authentication).







% Zeek as IDS
%https://docs.zeek.org/en/v3.0.14/scripts/policy/frameworks/files/detect-MHR.zeek.html


% defer to existing literature on detecting anomalies with zeek
% https://medium.com/hootsuite-engineering/anomaly-based-intrusion-detection-system-using-machine-learning-a18e88694ce0


%Anomaly detection:
%FlowLens: Enabling Efficient Flow Classification for ML-based Network Security Applications





\paragraph{Expected outcome} We will develop an open-source dashboard to help researchers visualize network flows and manage allow lists on their own, such that the dashboard can be integrated into a generic network architecture, apply rules to the control plane, and manipulate the routing of network flows on the slow and fast paths. Also, we will develop a mechanism to catch user errors by checking against the output of Task 1 and running anomaly detection algorithms in the fast path.


\paragraph{Lead investigator} PI Huang and co-PI Cappos will lead the effort to implement the system. Cappos, in particular, has extensive experience developing and deploying large-scale open-source software. All components in our design use open-source libraries, including the dashboard (based on Python Flask), traffic measurement (based on Zeek, whose logs are written to a Spark cluster on which to run Isolation Forest). Co-PI Pahle will assist in testing the prototype and integrating with NYU's existing HSRN infrastructure.
