\begin{center}
{\large \bf \TITLE}
\end{center}

\begin{center}
    {\bf Data Management Plan}
\end{center}






\noindent {\bf Data Types:} Several types of data will be generated by this project:

\begin{enumerate}
\item Information about data artifacts,
\item Specifications and code for sys,
\item Technical reports and scientific articles
\end{enumerate}

The first item is all public information we are retrieving and collating, so
we will keep the data we gather public.  The second item will be generated
by our team (as we write the software) and will be done in a public GitHub
repository.  The third item(s) will be generated initially privately for
submission to potentially anonymous venues but then published afterwards.  We
do not anticipate any data we use in this research will be private and we
will make it all available for use by researchers.


%While constructing \sysname, we will gather enough information
%handle any abuse complaints or similar issues.   To this end, we gather and
%collect flow information in a similar manner to PlanetLab's PlanetFlow
%service.
%We also log various types of experimenter use for our testbed for
%debugging and software improvement purposes.
%We also will log information to allow us to understand the size and diversity
%of our user base.   We log sufficient information to improve and understand
%how developers are using our clearinghouse and our nodes.
%
%As discussed in the proposal, the Android app for \sysname itself
%has privacy and data management protections built throughout the technical
%design.  Uses of sensor data
%must pass through the experimenter's IRB procedure as was already approved by
%an extensive IRB review at our institution.
%However, non-sensitive access (such as for experiments
%that do not specifically want sensor data)
%will proceed without undue restrictions.   The security layer functionality
%enables us to write customized access control and data flow restrictions for
%researcher experiments as is needed.


%The experimenters that use our testbed will be responsible for generating and
%managing their own data.   This will be done according to the policy set out
%by their data management plan / IRB.


\noindent {\bf Legal and Ethical Issues:} Due to adherence to IRB policies, no specific
ethical or legal issues are anticipated to arise in this work.
The PIs (from NYU, Purdue, and NJIT) will request IRB approval in the case
where user focused data collection is desirable. In addition, the data
generated through this research is not expected to be covered by any
copyright or database right.

%Our testbed restricts the set of
%things a researcher can do to prevent things like source address spoofing
%and ICMP packets (both are major sources of complaints on PlanetLab).
%As we will utilize some of the functionality from the Seattle testbed, we
%will also port their emergency stop mechanism over.  In the case where there
%is a complaint of abuse, we can use this mechanism and then contact the
%experimenters to understand more about whether the experimenter use was
%appropriate or not.

\vspace{5pt}
\noindent {\bf Access, Data Sharing, Reuse:} All of the code from this project is and
will continue to be released under a permissive open source license such
as Apache 2.0 or MIT.
The project's website is hosted on GitHub and provides access to the prototype
implementation, documentation and system specification.
The source code is hosted on a public GitHub repository.
Project results will be widely
disseminated to the community through project website, conference publications,
journal articles, NSF annual meetings, annual reports, and similar methods.

Dissemination to general public will be achieved through media
outlets, magazine publication, and presentation in public forums.
Broad discussion of this platform is key to growing and diversifying our
user base.   The PI has cultivated relationships with print and television
journalists that has broadened the visibility of other projects.   These
relationships will be utilized here to evangelize and grow the user base of
XXX.

The data and metadata collected about software is provided voluntarily by the
entities providing that software and will be used to determine the
trustworthiness of the software that is made available to the end user.

%We will build a web-based version for the tool that will be used by project owners to design and
%generate supply chain layouts. The tool will capture usage information
%when creating supply chain layouts.
%We will also request feedback from users about the usability and features
%available in the tool. All of this information will be used internally by the project team
%to improve the usability of the layout creation tool and will not be shared with any other parties.
%Users of the tool will be informed about what kind of of information
%is collected, for what purpose it is being used, and will be asked for consent before using the tool.

\vspace{5pt}
\noindent {\bf Data Standards and Capture Methods:}
As stated in the proposal, Observers will capture information.  In the
case that it is desirable, user feedback will be collected
using a standard web-based user
interface.
%Data will be collected from
%user studies via a standard web-based user interface.
%node manager and security
%layer / reference monitor interposition in experiments.

We will continue to use standard development tools like GitHub to manage
code, document the project, and synchronize effort across developers.


\vspace{5pt}
\noindent {\bf Short-term and Long-term Data Storage and Data Management:}
To date, the amount of data stored by this work can fit on a standard
server.  Although the anticipated data volume is hard to evaluate, we
estimate that a 2TB drive will suffice.

No specific security measures will be needed
concerning data storage. In general, storage will follow the procedure used in
Professor Cappos's Secure Systems Lab at NYU.

\begin{enumerate}
\item System and measurement data will be stored and kept online for three
years beyond the project life.
\item Source code will be stored on both the workstations and servers. Git
is an adequate tool to achieve this long-term archiving task.
\item Electronic copies of reports and articles will be kept on multiple
devices, such as cloud document storage, laptops, desktops, and servers.
\end{enumerate}

\vspace{5pt}
\noindent \textbf{Authorization for data access and protection of data:}
In the case where we do elect to conduct a user study,
user data will be stored in a format in which personal
identifiable information such as names will be replaced
by a user ID to protect user privacy.
Volunteer participation and user study data will be kept
in a secure location as per the IRB
instructions.
After the data
collection phase, the data will be moved into storage on
machines that have limited connectivity to the Internet, in order to
minimize the risk of data theft.


\vspace{5pt}
\noindent {\bf Resources:}
The PIs will supervise and be responsible for implementing and maintaining this
data management plan.  Senior personnel, students, and open source contributors
will be given a briefing on this plan prior to obtaining write access to the project's
GitHub repository. This will ensure a smooth implementation of the plan.




















