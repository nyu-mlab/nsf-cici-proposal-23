\begin{center}
{\large \bf \TITLE}
\end{center}

\begin{center}
    {\bf Data Management Plan}
\end{center}


This proposed project has strong support from NYU IT Services, which will provide us with: staff support; dedicated High-Speed Research Network (HSRN) and NYSERNET/I2 access; High Performance Computing (HPC) resources (Greene, Hudson, and Peel clusters and NSF Nautilus Computer and Information Science and Engineering (CISE) CommunityResearch Infrastructure (CCRI) initiative computer CHASE-CI 8-GPU systems, dedicated server and storage capacity in the NYU Research Computing Data Center (RCDC); data back-up and archiving; and other needed support to ensure effective and secure project data management.

\paragraph{Data Management and Storage}
Proposed project activities will use an open access data management approach. Where possible open source software, APIs, applications, and interfaces will be used and integrated to maximize interoperability and sharing of the open-source tools that are developed. The project will leverage an open-source data sharing and integration platform at GitHub (especially for sharing with ex-NYU parties and for community building) and at NYU to facilitate data capture, exchange, and backup, and provide a simple user interface to combine complex data with visualization and analysis tools. The project also incorporates an interactive computing infrastructure integrating sensors and adaptive software engineering systems to facilitate collaborative and adaptive education and physical rehabilitation, including use of AR/VR/MR and mobile devices.

Design requirements that will be implemented to support reliable data interchange, storage and management include:

\begin{itemize}
    \item Data is easily discoverable for researcher
    \item Data management complies and enforces meta-data standards
    \item Data management allows for data provenance and automatic versioning
    \item Real time data acquisition and distribution
    Advanced learner models coupled with visual analytic capabilities
\end{itemize}

The project team will use distributed restful (Representational State Transfer [REST]) data services to support internet access and allow service chaining with data processing and analytical services, as well as direct use for data and motion capture, visualization, sound and haptic interactions. This structure allows collaborators to create data/analytical workflows that can be validated, reproduced, and shared. Depending on available components, these workflows can be configured to integrate and fuse data with sensor output, analytical results, or with other sources. Within this framework, each researcher/partner producing data will be responsible for compliance with the standards, and for management to keep data up to date, current, secure and accessible to the whole research group, while limiting unnecessary network transfers. Metadata for each dataset (and version) will be automatically generated and entered into a local dataset repository that supports external search and query. The goal is to give researchers the ability to discover datasets across all research sets and sites. The project will also provide a convenience service to automatically harvest all participant data into the centralized repository for management, backup, and facilitate discovery.

Passwords and encryption will be provided as necessary to comply with applicable NYU and NSF guidelines. We will employ a central password store that supports all partners, providing secure file transfers via https and on-disk encryption. Redundant, fast, centralized data storage with off-site backup will be supported by NYU- IT services. Co-location of individual data repositories on the hardware will minimize network transfers and potential traffic delays. The project team will provide open access data wherever possible.

\paragraph{Policies for Access and Sharing}
In accordance with NYU’s Internal Review Board (IRB) policies, in each case involving Human Subjects data, each member of the investigative team, their collaborators, and all researchers with access to Human Subjects data will receive instruction and certification in Responsible Conduct of Research (RCR) and will comply with related Human Subjects and IRB requirements.

Data will be made available for sharing to qualified parties by the PIs upon request, unless such a request potentially compromises intellectual property interests, interferes with publication, invades subject privacy, betrays confidentiality, or precedes data curation. Shared data will include standards and notations needed to interpret the data, following commonly accepted practices in the field. Data will be available for open access and sharing as soon as is reasonably possible, normally no longer than 18 months after its acquisition.

To the full extent allowable by the NYU Office of Research and Integrity, data and meta-data collected from our studies will be publicly available for download, which will be accompanied by a report of our conclusions from the data, and recommendations regarding generalizability. A clickable disclosure agreement will be required before downloading any data, which guarantees that the data ca is available only for research use and cannot be used for commercial purposes. Data will be retained for at least three years beyond the award period, as required by NSF guidelines.

In the event that discoveries or inventions are made in direct connection with this data, access will be granted upon request once appropriate invention disclosures and/or provisional patent filings are made. Key data relevant to the discovery will be preserved until all issues of intellectual property are resolved. The data acquired and preserved in the context of this proposal will be further governed by the participating institutions' policies pertaining to intellectual property, record retention, and data management. Papers, technical reports, press and journal articles will also be accessible via our web sites at the respective institutions, with links to the publishers and abstracts and summaries of the work.

