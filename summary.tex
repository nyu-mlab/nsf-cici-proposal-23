\begin{center}
{\large \bf \TITLE}
\end{center}

\fp {\bf Overvew:}

Many academic institutions have high speed research networks (HSRNs) that support up to 400 Gbps, enabling high bandwidth data transfers (e.g., sending large datasets within and across institutions) and low latency applications (e.g., AR/VR). Some of the data and code could be sensitive in nature. To prevent unwanted data exfiltration and theft of intellectual property, network administrators typically use off-the-shelf commercial or open-source security products. However, products like intrusion detection systems (IDS) may alert after some delay; on our 400 Gbps network, every second of delays in alerts, for instance, could result in 50 GB of data exfiltrated. Furthermore, products like firewalls and intrusion prevention systems (IPS) inspect traffic in-line — generally fast enough for traditional networks (e.g., 10 Gbps), but are too slow (or prohibitively expensive) for the 400 Gbps HSRN in our institution, restricting the bandwidth and introducing additional latency.

Our goal is to maintain the bandwidth and latency of HSRNs (>400 Gbps) while minimizing security risks, including unwanted data exfiltration and theft of intellectual property on the HSRNs. To this end, we propose to develop and deploy a new open-source system to address this issue using a researcher-in-the -loop approach to manage connections based on allow-lists on software-defined networks (SDNs), in a way that is usable and minimizes overhead to the HSRN. Using this approach, we will first evaluate and pilot this system within NYU's HSRN, and evaluate it with HSRN researchers and network administrators in terms of usability and performance, before working to deploy it with collaborators in the broader HSRN ecosystem.

The figure on the right illustrates our proposed technique. A researcher (e.g., who needs to transfer large files and/or who experiments with AR/VR) connects their computer [A] to the HSRN. All traffic to/from [A] goes through an SDN switch [B] (e.g., SONIC), which by default forwards the traffic to traditional in-line security products like firewalls and IPS devices [C]. This slow path is meant to handle traditional traffic such as OS updates and web browsing on [A]. Any system compromises on the HSRN will likely be detected on this slow path. When the researcher decides to utilize the high bandwidth and low latency of the HSRN, they visit a custom-built researcher-facing dashboard [D] to proactively indicate their intended destination (e.g., another computer running an experimental AR/VR application) or retroactively select an existing connection (e.g., an established data transfer to a Dept of Energy server) to be added to a temporary allow-list. The SDN controller [E] (e.g., based on the P4 Integrated Network Stack) implements this allow-list as flow-based rules on the switch [B], which then forwards the traffic via switch [F] to the fast path. To minimize human errors on the dashboard, switch [F] mirrors traffic on the fast path to Zeek [G], an IDS which analyzes the first N packets of each flow (handshakes), samples the remaining packets (encrypted), and alerts the controller [E] of suspicious connections. To complement Zeek, we save all Zeek logs to a Spark cluster on which to identify further anomalies, e.g., using RADAR and Spot.

\fp {\bf Intellectual merit:}

(1) A usable method (e.g., implemented at [D]) to help non-experts to proactively and retroactively add destinations or connections to the temporary allow list, augmented with traffic visualization and automated security advice. (2) A technique to balance the trade-off between security and accuracy of alerts (e.g., implemented at [G]) for different experimental needs. (3) Evaluation on actual HSRN at NYU in terms of security and performance, with a special focus on usability.

\fp {\bf Broader impacts:}

An open-source and extensible system for general HSRNs, allowing HSRN researchers to enjoy the full high-bandwidth and low-latency capabilities while minimizing security risks. A teaching tool for the intersection of usable security and software-defined networking.


\fp {\bf Keywords:} usability, security, networking
