\begin{center}
{\large \bf \TITLE}
\end{center}

\fp {\bf Overview:}

Many academic institutions have high speed research networks (HSRNs) that support up to 400 Gbps connections to data centers and the cyber infrastructure. To enable high bandwidth transfers research networks typically rely on data transfer nodes in the DMZ. However, in our case we have built a separate and dedicated network that reaches into research labs and research offices at up to 200 Gbps with redundant connections. This allows participating researchers to conduct high bandwidth data transfers, such as sending large datasets within and across institutions, and to support  low latency research applications, such as AR/VR and robotics. Researchers from many disciplines are benefiting from these high speed low latency transfers. . Active researchers on our own HSRN include AR/VR, Physics, Chemistry, Robotics, Electrical Engineering, Multimedia and Performing Arts, and Computer Science.

On HSRNs, much of the data and code is potentially highly sensitive in nature, and can include personally identifiable information about experimental subjects, key experimental data that could be of interest to national security, and intellectual property such as code, research data and results, and papers.
An attacker could potentially access, exfiltrate, or disrupt (e.g., by modifying) sensitive research data and code on an HSRN or misuse the hardware connected to the network (e.g. DDNS). Such an  attack could originate from the Internet or from another host on the same research network (e.g., as a result of malware's lateral movement

We plan to develop, deploy, and evaluate a usable system that lets researchers temporarily and securely bypass security appliances (e.g., intrusion prevention systems, or IPS) for specific use cases — on their own with informed decisions. We propose developing a usable dashboard for researchers to specify their workloads and bypass security appliances. At the same time, we will use a fallback mechanism to detect anomalies and catch any mistakes made by the user (who is not an expert in network security). Our system will be hardware or infrastructure agnostic and will benefit high speed networks in general.

\fp {\bf Intellectual merit:}

(1) A usable method to help non-experts to proactively and retroactively add destinations or connections to the temporary allow list, augmented with traffic visualization and automated security advice. (2) A technique to balance the trade-off between security and accuracy of alerts  for different experimental needs. (3) Evaluation on actual HSRN at NYU in terms of security and performance, with a special focus on usability.

\fp {\bf Broader impacts:}

An open-source and extensible system for general HSRNs, allowing HSRN researchers to enjoy the full high-bandwidth and low-latency capabilities while minimizing security risks. A teaching tool for the intersection of usable security and software-defined networking.


\fp {\bf Keywords:} usability, security, networking
