\section{Results from Prior NSF Support}


\paragraph{Justin Cappos}
PI Cappos has a history of deploying software widely in practice and then
tackling the research problems that arise. PI Cappos has designed improved
security models for Linux package managers~\cite{Cappos_CCS_2008}, cloud
container systems~\cite{Kuppusamy_NSDI_2016, Kuppusamy_USENIX_2017}, and
version control systems~\cite{Torres_USENIXSec_2016} and in all cases
deployed these new models into use across millions of devices.

Due in large part to NSF support (1513457, 1444827, 1223588), PI Cappos's
research advances have been published in top venues in
security~\cite{Cappos_CCS_2010, Samuel_CCS_2010, Torres_USENIXSec_2016, Nikitin_USENIXSec_2017, in-toto-paper},
networking~\cite{Zhuang_NSDI_2014, Kuppusamy_NSDI_2016},
education~\cite{Cappos_SIGCSE_2009, Cappos_SIGCSE_2014, Hooshangi_SIGCSE_2015},
software engineering~\cite{Rasley_ISSRE_2015, Oliveira_ACSAC_2014,
Gopstein_FSE_2017},
and systems~\cite{Li_USENIX_2015, Li_USENIX_2017, Kuppusamy_USENIX_2017}.

The PIs all worked together on NSF grant 1801376 (\emph{Securing the Software Supply Chain},
2018-2021, \$1.2M)~\cite{Torres_USENIXSec_2016, le-git-imate, ifip-sec-19, in-toto-paper , le-git-imate-journal}, with PI Torres-Arias as the lead PhD student.
As part of this project, the PIs collaboratively had two major outcomes.  First they found serious design
vulnerabilities and in fixed them by created the tag signing model used by
{\tt git}.  They also created In-Toto~\cite{in-toto-website}, which has been adopted by the Linux
Foundation project and which is widely used in industry.
This has led to adoption of the software
written by the personnel across millions of devices.

\textbf{Broader Impacts:} PI Cappos emphasizes the transition of work into practical use and had
substantial success on that front.  For example, NSF
grant 1345049 ``NSF TTP: Securing Python Package Management with TUF''
provided a year of funding to transition TUF into use, leading to its
standardization by Python and use by
DigitalOcean, CloudFlare, DataDog, RedHat, VMware, Microsoft, Google,
IBM, Amazon, and Docker.
Overall, PI Cappos has
transitioned research into production use in a wide array of widely used
software including Git, Python, popular cloud environments, automobiles,
and most Linux distributions.

PI Cappos also has substantial experience with using his research in the
classroom.  His Seattle testbed (supported by NSF grants 1405904,
1405907, 1547290, and 1205415) has been used in about 100
security and networking classes.
He built a core community from the early adopters, through which
he developed a group of colleges that use the existing educational
materials~\cite{NWDCSD}.

The PIs have also used NSF support to increase impact with faculty at high
schools and non-research institutions.
To support the educational component for NSF grant 1054754,
PI Curtmola has organized a 4-day professional development workshop
for high school teachers~\cite{asee14,ccwt14}, helping them
develop curriculum units to be integrated into high-school subjects.

PI Cappos has been teaching a computer
security summer camp for college and high school faculty through NSF
awards (NSF 1241653: ``NSF DUE: Building Cyber Security Capacity in Two Year
and Four Year Colleges'' and NSF 1407161: ``NSF RET in Engineering and
Computer Science Site: Research Experience and Training in Cyber Security for
Pre-College Teachers'').  This has provided invaluable connections with
educators who use the PI's educational modules.  Such contacts provide a
natural mechanism that he has used to successfully disseminate educational
modules~\cite{Cappos_SIGCSE_2014, Hooshangi_SIGCSE_2015}.










\paragraph{Feamster}
\textbf{CPS: Medium: Detecting and Controlling Unwanted
Data Flows in the Internet of Things} \textit{1739809} (PI: Feamster);
10/1/18--9/30/22, \$875,000.
This
project develops technologies that ensure that IoT smart devices remain secure
and protect user privacy in the face of the widespread deployment of connected
smart devices.
\textbf{Intellectual Merit:} This project advances the theory and practice of network traffic
analysis and anomaly detection to secure IoT deployments. The project
has resulted in publications at top-tier
conferences~\cite{hooman2019:ccs,chu2018:iot:iotj,zheng2018:iot:cscw,feamster2018:iot:ctlj,doshi2018:iot,apthorpe2018:iot:imwut,datta2018:iot,acar2018:iot,weinberg2019:iot:www}, including studies of how home network traffic can
expose users to new privacy risks, new methods for modeling user attitudes and
norms about
privacy in smart homes, with applications to using contextual integrity,
and a broader understanding of how smart home
devices from appliances to smart
TVs may
collect private user data. They also include broader research on IoT
security and
privacy~\cite{chu2018:iot:iotj,zheng2018:iot:cscw,feamster2018:iot:ctlj,doshi2018:iot,apthorpe2018:iot:imwut,datta2018:iot,acar2018:iot,weinberg2019:iot:www}.
\textbf{Broader Impacts:} The research
has produced open-source software (IoT Inspector) to help smart home users better understand
how devices collect and share private data and the largest
labeled dataset of smart home device traffic with data from about 10,000
homes.

