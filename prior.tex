\section{Results from Prior NSF Support}

\subsection*{Danny Yuxing Huang}
PI Huang is supported by NSF Award 2219867 ``Privacy-preserving IoT Analytics and Behavior Prediction on Network Edge.'' This has given him the experience of analyzing massive amount of network traffic and conducting machine learning, such as anomaly detection, on network flows.


\subsection*{Robert Pahle}
Co-PI Pahle is supported by NSF Award 2232817 ``DataStorage: Distributed, Fast, Scalable Infrastructure for Emerging Media Research Data.'' This has given him extensive experience in conducting experiments on NYU's high-speed research network infrastructure and collecting preliminary data on workloads and user experience.



\subsection*{Justin Cappos}
Co-PI Cappos has a history of deploying software widely in practice and then
tackling the research problems that arise. PI Cappos has designed improved
security models for Linux package managers~\cite{Cappos_CCS_2008}, cloud
container systems~\cite{Kuppusamy_NSDI_2016, Kuppusamy_USENIX_2017}, and
version control systems~\cite{Torres_USENIXSec_2016} and in all cases
deployed these new models into use across millions of devices.

Due in large part to NSF support (1513457, 1444827, 1223588), PI Cappos's
research advances have been published in top venues in
security~\cite{Cappos_CCS_2010, Samuel_CCS_2010, Torres_USENIXSec_2016, Nikitin_USENIXSec_2017, in-toto-paper},
networking~\cite{Zhuang_NSDI_2014, Kuppusamy_NSDI_2016},
education~\cite{Cappos_SIGCSE_2009, Cappos_SIGCSE_2014, Hooshangi_SIGCSE_2015},
software engineering~\cite{Rasley_ISSRE_2015, Oliveira_ACSAC_2014,
Gopstein_FSE_2017},
and systems~\cite{Li_USENIX_2015, Li_USENIX_2017, Kuppusamy_USENIX_2017}.

\textbf{Broader Impacts:} PI Cappos emphasizes the transition of work into practical use and had
substantial success on that front.  For example, NSF
grant 1345049 ``NSF TTP: Securing Python Package Management with TUF''
provided a year of funding to transition TUF into use, leading to its
standardization by Python and use by
DigitalOcean, CloudFlare, DataDog, RedHat, VMware, Microsoft, Google,
IBM, Amazon, and Docker.
Overall, PI Cappos has
transitioned research into production use in a wide array of widely used
software including Git, Python, popular cloud environments, automobiles,
and most Linux distributions.

PI Cappos also has substantial experience with using his research in the
classroom.  His Seattle testbed (supported by NSF grants 1405904,
1405907, 1547290, and 1205415) has been used in about 100
security and networking classes.
He built a core community from the early adopters, through which
he developed a group of colleges that use the existing educational
materials~\cite{NWDCSD}.

The PIs have also used NSF support to increase impact with faculty at high
schools and non-research institutions.
To support the educational component for NSF grant 1054754,
PI Curtmola has organized a 4-day professional development workshop
for high school teachers~\cite{asee14,ccwt14}, helping them
develop curriculum units to be integrated into high-school subjects.

PI Cappos has been teaching a computer
security summer camp for college and high school faculty through NSF
awards (NSF 1241653: ``NSF DUE: Building Cyber Security Capacity in Two Year
and Four Year Colleges'' and NSF 1407161: ``NSF RET in Engineering and
Computer Science Site: Research Experience and Training in Cyber Security for
Pre-College Teachers'').  This has provided invaluable connections with
educators who use the PI's educational modules.  Such contacts provide a
natural mechanism that he has used to successfully disseminate educational
modules~\cite{Cappos_SIGCSE_2014, Hooshangi_SIGCSE_2015}.










