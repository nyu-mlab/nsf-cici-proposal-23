\section{Intellectual Merits}

Our key intellectual merit is usability. The process of managing allow lists has traditionally been the work of network administrators in general, but the process could be manual and error prone. Our system transfers the responsibility to researchers for efficiency and flexibility. We help the researchers, who are not experts in network security, make informed decisions through annotations of flows. We catch errors by comparing against typical workloads of researchers through an anomaly detection algorithm (Isolation Forest), which we improve, via reinforcement learning, based on feedback from researchers. All these steps will allow researchers to enjoy the high bandwidth and low latency of an HSRN, have the flexibility to add workload without the barriers of network administrators, and get protected from the allow lists and a backup anomaly detection to catch mistakes. Our design of the system is meant to be modular and generic. We expect our design to be hardware agnostic, so that it can be integrated into the NYU infrastructure and likely other institutions as well.
